% --- Core ---
\usepackage[utf8]{inputenc}
\usepackage[T1]{fontenc}
\usepackage[margin=2.5cm]{geometry}
\usepackage{enumitem}

% --- Math & Symbols ---
\usepackage{amsmath, amssymb, amsfonts, amsthm}
\usepackage{cancel}

% --- UI & Layout ---
\usepackage{parskip} % Spasi antar paragraf, tanpa indentasi
\usepackage{float}
\usepackage{graphicx}
\usepackage{caption}
\usepackage{subcaption} % Untuk sub-figures
\usepackage{booktabs}   % Tabel cantik ( \toprule, \midrule, \bottomrule )
\usepackage{array}
\usepackage{lipsum}

% --- Colors & Links ---
\usepackage[dvipsnames]{xcolor}
\usepackage{hyperref}

% --- Code Snippets (Listings) ---
\usepackage{listings}
\definecolor{codegreen}{rgb}{0,0.6,0}
\definecolor{codegray}{rgb}{0.5,0.5,0.5}
\definecolor{codepurple}{rgb}{0.58,0,0.82}
\definecolor{backcolour}{rgb}{0.95,0.95,0.92}

\lstset{
    backgroundcolor=\color{backcolour},   
    commentstyle=\color{codegreen},
    keywordstyle=\color{magenta},
    numberstyle=\tiny\color{codegray},
    stringstyle=\color{codepurple},
    basicstyle=\ttfamily\small,
    breakatwhitespace=false,         
    breaklines=true,                 
    captionpos=b,                    
    keepspaces=true,                 
    numbers=left,                    
    numbersep=5pt,                  
    showspaces=false,                
    showstringspaces=false,
    showtabs=false,                  
    tabsize=2
}

% --- Diagrams (TikZ) ---
\usepackage{tikz}
\usepackage{tikz-network}
\usetikzlibrary{trees, graphs, arrows.meta, shapes.geometric, positioning}

% --- Boxes (tcolorbox) ---
\usepackage[breakable]{tcolorbox}
\tcbuselibrary{theorems}

\newtcbtheorem[number within=chapter]{definisi}{Definisi}%
{breakable, colback=green!5,colframe=green!35!black,fonttitle=\bfseries}{def}

\newtcbtheorem[number within=chapter]{teorema}{Teorema}%
{breakable, colback=blue!5,colframe=blue!35!black,fonttitle=\bfseries}{th}

\newtcbtheorem[number within=chapter]{contoh}{Contoh}%
{breakable, colback=red!5,colframe=red!35!black,fonttitle=\bfseries}{ex}

% --- Header & Footer ---
\usepackage{fancyhdr}
\pagestyle{fancy}
\fancyhf{}
\fancyhead[L]{\small IF4031 - JagaWarga}
\fancyhead[R]{\small \leftmark}
\fancyfoot[C]{\thepage}

% --- Referencing ---
\usepackage[noabbrev, capitalize]{cleveref}
\graphicspath{{images/}}

\renewcommand{\contentsname}{Daftar Isi}
\renewcommand{\chaptername}{Bab}
\renewcommand{\figurename}{Gambar}
\renewcommand{\tablename}{Tabel}
\renewcommand{\bibname}{Daftar Pustaka} % Untuk document class report/book